\section{Introduzione}

L'idea per il progetto di questa tesi nasce come risposta a un problema sempre più attuale in un mondo sempre più connesso.
La molteplicità di contatti, la velocità delle comunicazioni e l'accesso universale alle notizie 
rendono la creazione, l'organizzazione e la partecipazione ad eventi estremamente semplice.
Pensiamo alle riunioni di lavoro, alle serate tra amici, agli appuntamenti per un caffè.
Ma anche a una fiera, una convention aziendale, ad un concerto, alla partita di calcio o alla mostra dell'artista che ci è sempre piaciuto.\\
\hfill \break
A volte siamo noi a proporre, a volte ci invitano. \\
Quando ci invitano, spesso magari abbiamo già un altro impegno, o magari un invito di un altro contatto a cui dobbiamo ancora dare conferma. 
E sul momento magari non ci si ricorda, si conferma per poi dover, purtroppo, disdire l'evento sovrapposto.\\
Quando siamo noi a proporre, potremmo trovarci nella difficoltà di cercare un evento da proporre, 
scrutando centinaia di profili social di tutti i locali di cui abbiamo sentito parlare nella speranza che propongano qualcosa, 
oppure non sappiamo se l'altra persona possa essere libera o meno. Ancora peggio negli eventi di gruppo, in cui bisogna riuscire a far combaciare gli impegni di tre, quattro o più persone.\\
\\
Si è rivelata l'opportunità di usare gli attuali strumenti tecnologici per la creazione di un programma che permetta 
di gestire gli eventi sia a livello di proposte, che a livello di conferme.\\
\\
Alla base della funzionalità sussiste l'idea di affiancare alla classica agenda con gli impegni presi(e quindi confermati) 
un'altro calendario in cui sono presenti tutti gli eventi a cui potremmo partecipare. La conferma di un evento sposterà l'evento all'interno dell'agenda.\\
Gli eventi creati potranno essere condivisi a persone o gruppi di persone, e potremo vedere chi conferma la sua presenza.\\
Inoltre, nella società delle immagini e della condivisione, per sopperire a chi di foto ne fa poche, e per aiutare a mantenere ordinato l'archivio, c'è la possibilità di condividere con chiunque abbia partecipato all'evento.\\
\\
La realizzazione di questo tipo di programma prevede particolarità che incrociano tante necessità diverse.
In primis la persistenza dell'agenda dell'utente, che deve essere salvata e aggiornata per garantire affidabilità e coerenza su più device.
Si aggiunge poi l'aspetto della condivisione degli eventi, che vede necessario l'aggiornare ogni singolo device interessato online con le modifiche apportate.
Infine, il caricamento delle foto introduce la gestione di richieste e persistenza di dimensioni importanti.\\
\\
Tramite l'analisi dei requisiti e lo sviluppo del progetto verranno evidenziate le scelte tecnologiche che hanno portato all'implementazione di una applicazione affidabile, scalabile e utile.  


\newpage