\section{Introduzione}

L'idea per il progetto di questa tesi nasce come risposta a un problema sempre più attuale in un mondo sempre più connesso.
La molteplicità di contatti, la velocità delle comunicazioni e l'accesso universale alle notizie 
rendono la creazione, l'organizzazione e la partecipazione ad eventi estremamente semplice ma al contempo frenetico.
Si fa fatica a seguire a tutte le occasioni a cui potremmo prendere parte.
Pensiamo alle riunioni di lavoro, alle serate tra amici, agli appuntamenti per un caffè.
Ma anche a una fiera, una convention aziendale, ad un concerto, alla partita di calcio o alla mostra dell'artista che ci è sempre piaciuto e che passa per una volta nella città vicina.
Queste occasioni spesso si accavallano o si finisce per dimenticarsene, potenzialmente creando delusione e/o frustrazione.\\
\hfill \break
Quando condividiamo un evento, a volte siamo noi a proporre, altre volte ci invitano. \\
Quando ci invitano, spesso magari abbiamo già un altro impegno, o magari un invito di un altro contatto a cui dobbiamo ancora dare conferma. 
E sul momento magari non ci si ricorda, si conferma per poi dover, purtroppo, disdire l'evento sovrapposto.\\
Quando invece siamo noi a proporre, potremmo trovarci nella difficoltà di trovare un evento da proporre, 
scrutando centinaia di profili social di tutti i locali di cui abbiamo sentito parlare nella speranza che propongano qualcosa, 
oppure non sappiamo se l'altra persona possa essere libera o meno. 
Questo problema si ripresenta ancora più grave nell'organizzazione di gruppo, in cui bisogna riuscire a far combaciare gli impegni di tre, quattro o più persone.\\
\\
Ecco quindi l'opportunità di creare uno strumento che permetta di semplificare la proposta e la gestione degli eventi, 
separando la proposta dalla conferma, per dare modo di valutare l'effettiva disponibilità ma anche rendere più facile un'invito a partecipare.
Allo stesso modo si può semplificare la ricerca di occasioni, creando uno spazio unico virtuale in cui pubblicare e consultare gli eventi.\\
\\
Alla base della funzionalità sussiste l'idea di affiancare alla classica agenda degli impegni presi (e quindi confermati) 
un'altro calendario in cui sono presenti tutti gli eventi a cui si potrebbe partecipare. 
La conferma di un evento lo sposterà all'interno dell'agenda.\\
Gli eventi creati potranno essere condivisi a persone o gruppi di persone, e sarà possibile vedere chi conferma la sua presenza.\\
Inoltre, nell'epoca delle immagini e della condivisione, si prevede la possibilità di condividere le proprie foto con chiunque abbia partecipato all'evento.\\
\\
La realizzazione di questo tipo di programma prevede particolarità che incrociano tante necessità diverse.
In primis la persistenza dell'agenda dell'utente, che deve essere mantenuta e aggiornata per garantire affidabilità e coerenza per un uso distribuito del servizio.
Si aggiunge poi l'aspetto della condivisione degli eventi, che vede necessario l'aggiornare tutti gli attori interessati dalle modifiche apportate.
Infine, il caricamento e salvataggio delle foto introduce la gestione di richieste e di memoria di dimensioni importanti.\\
\\
Tramite l'analisi dei requisiti e lo sviluppo del progetto verranno evidenziate le scelte tecnologiche che hanno portato all'implementazione di una applicazione efficace, affidabile e scalabile.  


\newpage