\section{Introduzione}

L'informatica svolge da anni ruoli sempre più essenziali nelle gestioni aziendali ma anche nella vita di tutti i giorni. 
Per applicazioni con particolari requisiti di sicurezza, migliaia di utenti o specifiche garanzie di servizio, la sicurezza, 
la scalabilità e l'affidabiliità non sono opzioni ma necessità. 
Tali funzionalità possono essere ottenute tramite infrastrutture progettate e costruite autonomamente, 
ma richiedono l'investimento di risorse importanti, basti pensare alla progettazione, al deployment su macchine fisiche e alla relativa manutenzione, sia fisica che logica.
Per fare un esempio, per poter garantire l'utilizzo anche in fase di richieste elevate, è necessario mantenere un'infrastruttura che statisticamente verrà usata in minima parte.
La maggior parte dei sistemi richiede operazioni che si discostano dalle caratteristiche centrali dell'applicazione, 
quali il monitoraggio degli eventi o tutta la gestione della sicurezza.\\
\\
I cloud providers nascono con la finalità di proporre piattaforme che risolvano la gran parte dei problemi comuni 
nascondendo e astraendo la complessità che questi problemi richiedono.
Questo comporta vantaggi per entrambi i lati: 
\begin{itemize}
    \item Lo sviluppatore può concentrarsi sulla logica applicativa, scegliendo il prodotto/i che più si addice alle sue esigenze, 
    preoccupandosi solamente di fare in modo che la configurazione dei servizi sia corretta. 
    Non deve più preoccuparsi per la gestione fisica delle macchine, e può, nella maggior parte dei casi, pagare solo per le risorse che utilizza.
    \item Il cloud provider, vendendo lo stesso prodotto a più clienti, concentra le risorse richieste per la manutenzione del servizio
    e ammorta i volumi computazionali eventualmente richiesti per la gestione di carichi elevati, 
    guadagnando sulle risorse risparmiate rispetto al caso in cui ogni servizio fosse stato gestito autonomamente.
\end{itemize}
    I cloud providers forniscono molteplici servizi con capacità e responsabilità diverse, specifiche per varie esigenze, 
avvicinandosi il più possibile ai bisogni specifici dei clienti.\\
\\
La facilità di configurazione e il costo ridotto iniziale dei servizi proposti rende possibile anche a realtà di piccole e medie dimensioni 
di creare progetti con capacità, ambizioni e qualità superiori a quelli che le loro normali risorse permetterebbero.
Per questa ragione, anche in fase di progetto e con i prototipi iniziali, conviene basarsi su risorse in cloud, 
integrando da subito funzionalità comunque eventualmente necessarie e 
identificando il più precocemente possibile gli strumenti più adatti all'applicazione che si sta costruendo.
Nella scelta dei servizi offerti dai cloud providers, si rivela facile confondersi tra le tante opportunità, 
che possono apparire in un primo giudizio molto simili ma, magari nate per scopi completamente differenti, 
implementate con architetture molto diverse, che ne determinano potenzialità e limiti.
Risulta quindi fondamentale saper individuare il servizio che più si addice alle proprie necessità, 
distinguendolo tra gli altri per le differenze essenziali che comporteranno un vantaggio nell'esecuzione del progetto.\\
\\
Tra i rischi maggiori che si corrono implementando un'applicazione tramite infrastrutture in cloud,
oltre al perdere il controllo del budget dati i costi variabili, sussiste la scelta sbagliata dei servizi da sfruttare, che, 
magari illundendo inizialmente un corretto funzionamento, può far sorgere problemi di integrazione o di funzionalità 
più avanti nella vita del prodotto.
Per quanto si presentino come soluzioni indipendenti e virtualizzate, 
la scelta sbagliata di un componente può comportare la riscrittura di parti intere del programma,
dal momento che ogni risorsa richiede un approccio differente.
La scelta corretta di un componente può avvenire solo avendo ben chiare le necessità architetturali e le particolarità del prodotto che si vuole implementare.\\
\\
L'individuazione di suddette necessità e particolarità deve avvenire congiuntamente all'analisi richiesta per sviluppare il codice.
Oltre a trovare lo strumento offerto più inerente alle proprie esigenze, 
bisogna che il codice implementato su tali strumenti risponda alle potenzialità e alle capacità che sono in grado di offrire, 
adattando le tecnologie alla soluzione ricercata.
Una struttura sicura, scalabile ed affidabile, infatti, lo è tanto grazie alle tecnologie usate quanto alle scelte ingegneristiche di come usarle.
Seguendo lo stesso approccio applicato nell'ingegneria del software, sarà possibile affiancare alla normale progettazione dello sviluppo 
la scelta e la modalità di utilizzo dei servizi in cloud.\\
\\
Nell'ingegneria del software, la branca che si occupa di sviluppare un prodotto partendo da un'idea iniziale, 
si individuano diverse fasi per la creazione di un software resistente e mantenibile, oltre che efficace.\\
\begin{itemize}
    \item La prima fase consiste con l'abstract, in cui si sintetizza l'idea generale del progetto, 
    specificando le funzionalità principali e la visione d'insieme dell'applicazione.
    \item Segue il documento dei requisiti, che analizza l'abstract e ne estrae in maniera formale i requisiti e introducendo i casi d'uso, 
    ovvero tutte le azioni che il programma può compiere.\\
    L'analisi del problema deduce una struttura iniziale e inizia a definire il comportamento generale del programma.\\
    Il documento dei requisiti e l'analisi vengono elaborati in collaborazione con il committente, 
    per assicurarsi che le richieste siano uniformemente intese da entrambe le parti.
    \item In fase di progettazione vengono prese decisioni ad alto livello indipendenti dalle tecnologie specifiche da utilizzare, 
    identificandone però le caratteristiche necessarie. 
    In questa fase si individuano i possibili punti critici e le particolarità richieste al sistema.
    Si definiscono quindi il tipo di architettura, la struttura del sistema e la sua interazione tra le parti.
    \item La fase di implementazione documenta le scelte applicate sia a livello tecnologico che a livello di realizzazione. 
    Seguendo le scelte prese in fase di progettazione, dettaglia le scelte architetturali, delle diverse partei del codice e della loro interzione.
\end{itemize}
Seguendo lo stesso schema, per lo sviluppo di un sistema che presenti nei requisiti l'essere scalabile ed affidabile, 
le scelte relative necessarie emergono, vendono analizzate e applicate in linea con le altre scelte progettuali del programma.\\
\clearpage











\subsection{Il progetto}

L'idea per il progetto di questa tesi nasce come risposta a un problema sempre più attuale in un mondo sempre più connesso.
La molteplicità di contatti, la velocità delle comunicazioni e l'accesso universale alle notizie 
rendono la creazione, l'organizzazione e la partecipazione ad eventi estremamente semplice ma al contempo frenetico.
Si fa fatica a seguire a tutte le occasioni a cui potremmo prendere parte.
Pensiamo alle riunioni di lavoro, alle serate tra amici, agli appuntamenti per un caffè.
Ma anche a una fiera, una convention aziendale, ad un concerto, alla partita di calcio o alla mostra dell'artista che ci è sempre piaciuto e che passa per una volta nella città vicina.
Queste occasioni spesso si accavallano o si finisce per dimenticarsene, potenzialmente creando delusione e/o frustrazione.\\
\\
Quando condividiamo un evento, a volte siamo noi a proporre, altre volte ci invitano. \\
Quando ci invitano, spesso magari abbiamo già un altro impegno, o magari un invito di un altro contatto a cui dobbiamo ancora dare conferma. 
E sul momento magari non ci si ricorda, si conferma per poi dover, purtroppo, disdire l'evento sovrapposto.\\
Quando invece siamo noi a proporre, potremmo trovarci nella difficoltà di trovare un evento da proporre, 
scrutando centinaia di profili social di tutti i locali di cui abbiamo sentito parlare nella speranza che propongano qualcosa, 
oppure non sappiamo se l'altra persona possa essere libera o meno. 
Questo problema si ripresenta ancora più grave nell'organizzazione di gruppo, in cui bisogna riuscire a far combaciare gli impegni di tre, quattro o più persone.\\
\\
Ecco quindi l'opportunità di creare uno strumento che permetta di semplificare la proposta e la gestione degli eventi, 
separando la proposta dalla conferma, per dare modo di valutare l'effettiva disponibilità ma anche rendere più facile un'invito a partecipare.
Allo stesso modo si può semplificare la ricerca di occasioni, creando uno spazio unico virtuale in cui pubblicare e consultare gli eventi.\\
\\
Alla base della funzionalità sussiste l'idea di affiancare alla classica agenda degli impegni presi (e quindi confermati) 
un'altro calendario in cui sono presenti tutti gli eventi a cui si potrebbe partecipare. 
La conferma di un evento lo sposterà all'interno dell'agenda.\\
Gli eventi creati potranno essere condivisi a persone o gruppi di persone, e sarà possibile vedere chi conferma la sua presenza.\\
Inoltre, nell'epoca delle immagini e della condivisione, si prevede la possibilità di condividere le proprie foto con chiunque abbia partecipato all'evento.\\
\\
La realizzazione di questo tipo di programma prevede particolarità che incrociano tante necessità diverse.
In primis la persistenza dell'agenda dell'utente, che deve essere mantenuta e aggiornata per garantire affidabilità e coerenza per un uso distribuito del servizio.
Si aggiunge poi l'aspetto della condivisione degli eventi, che vede necessario l'aggiornare tutti gli attori interessati dalle modifiche apportate.
Infine, il caricamento e salvataggio delle foto introduce la gestione di richieste e di memoria di dimensioni importanti.\\
\\
\clearpage

\subsection{Abstract del progetto}

Wyd è un'applicazione che permette ai clienti di organizzare i propri impegni, siano essi confermati oppure proposti.\newline
Mette a disposizione due calendari, il primo con gli eventi in cui l'utente è convinto di partecipare, 
il secondo in cui vengono riuniti gli eventi a cui l'utente è stato invitato ma senza aver ancora dato conferma.\newline
L'utente ha la possibilità di creare, modificare, confermare o disdire un evento, ma anche condividerlo con altri o allegarci foto.
La condivisione di un evento può avvenire con applicazioni esterne tramite la generazione di un link o grazie all'ausilio di gruppi di profili.
Inoltre, al termine di un evento, l'applicazione carica automaticamente le foto scattate durante l'evento, per allegarle a seguito della conferma dell'utente.\newline  
L'utente può infatti cercare altri profili e creare gruppi con i profili trovati.\newline
Tutta l'interazione avviene tramite l'utilizzo di profili, che permettono di suddividere semanticamente gli eventi e le relazioni.\newline

\newpage