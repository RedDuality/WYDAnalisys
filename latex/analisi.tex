\newpage
\section{Analisi del Problema}
\subsection{Analisi Documento dei Requisiti: Analisi delle Funzionalità}
\hfill \break

\textbf{Tabella delle Funzionalità}
\hfill \break

\begin{tabular} {|P{4cm}|P{4cm}|P{3.5cm}|P{4cm}|} % Qua cambiate a piacimento la larghezza
    \hline
    \textbf{Funzionalità} & \textbf{Tipo}                                                 & \textbf{Grado di complessità} & \textbf{Requisiti Collegati}                    \\
    \hline
    Login                 & Interazione esterno e lettura dati                            & semplice                      & R2F                                             \\
    \hline
    Registrazione         & Interazione esterno e memorizzazione dati                     & semplice                      & R1F                                             \\
    \hline
    EventiConfermati      & Interazione esterno e gestione dati                           & complessa                     & R4F, R8F                                        \\
    \hline
    EventiProposti        & Interazione esterno e gestione dati                           & complessa                     & R5F, R9F                                        \\
    \hline
    GestioneGruppi        & Interazione esterno e gestione dati                           & complessa                     & R13F, R14F                                      \\
    \hline
    GestioneProfili       & Interazione esterno e gestione dati                           & complessa                     & R10F, R12F                                      \\
    \hline
    VisualizzaEvento      & Interazione esterno e gestione, lettura e memorizzazione dati & complessa                     & R5F, R6F, R7F, R8F, R9F, R14F, R15F, R18F, R20F \\
    \hline
    AggiornaEvento        & Gestione dati                                                 & complessa                     & R17F                                            \\
    \hline
    RecuperaImmagini      & Lettura dati                                                  & complessa                     & R19F                                            \\
    \hline
    ScritturaLog          & Memorizzazione dati                                           & semplice                      & R21F                                            \\
    \hline
\end{tabular}

\hfill \break

\textbf{Registrazione: Tabella Informazioni/Flusso}
\hfill \break

\begin{tabular} {|P{3cm}|P{3cm}|P{3cm}|P{3cm}|P{3cm}|}
    \hline
    \textbf{Informazione} & \textbf{Tipo} & \textbf{Livello protezione/privacy} & \textbf{Input/Output} & \textbf{Vincoli}                                  \\
    Email                 & semplice      & Protezione alta                     & Input                 & Deve essere di 256 caratteri e del formato giusto \\
    \hline
    Password              & semplice      & Protezione molto alta               & Input                 & Deve essere almeno di 8 caratteri                 \\
    \hline
\end{tabular}
\hfill \break

\textbf{Login: Tabella Informazioni/Flusso}
\hfill \break

\begin{tabular} {|P{3cm}|P{3cm}|P{3cm}|P{3cm}|P{3cm}|}
    \hline
    \textbf{Informazione} & \textbf{Tipo} & \textbf{Livello protezione/privacy} & \textbf{Input/Output} & \textbf{Vincoli}         \\
    \hline
    Email                 & semplice      & Protezione molto alta               & Input                 & Non più di 256 caratteri \\
    \hline
    Password              & semplice      & Protezione molto alta               & Input                 & Non più di 50 caratteri  \\
    \hline
\end{tabular}
\hfill \break

\textbf{EventiConfermati: Tabella Informazioni/Flusso}
\hfill \break

\begin{tabular} {|P{3cm}|P{3cm}|P{3cm}|P{3cm}|P{3cm}|}
    \hline
    \textbf{Informazione}   & \textbf{Tipo} & \textbf{Livello protezione/privacy} & \textbf{Input / Output} & \textbf{Vincoli} \\
    \hline
    Lista Eventi Confermati & Composto      & Protezione media                    & Output                  &                  \\
    \hline
\end{tabular}
\hfill \break

\textbf{EventiProposti: Tabella Informazioni/Flusso}
\hfill \break

\begin{tabular} {|P{3cm}|P{3cm}|P{3cm}|P{3cm}|P{3cm}|}
    \hline
    \textbf{Informazione} & \textbf{Tipo} & \textbf{Livello protezione/privacy} & \textbf{Input / Output} & \textbf{Vincoli} \\
    \hline
    Lista Eventi Proposti & Composto      & Protezione media                    & Output                  &                  \\
    \hline
\end{tabular}
\hfill \break

\textbf{GestioneGruppi: Tabella Informazioni/Flusso}
\hfill \break

\begin{tabular} {|P{3cm}|P{3cm}|P{3cm}|P{3cm}|P{3cm}|}
    \hline
    \textbf{Informazione} & \textbf{Tipo} & \textbf{Livello protezione/privacy} & \textbf{Input / Output} & \textbf{Vincoli}     \\
    \hline
    Elenco Gruppi         & Composto      & Protezione media                    & Output                  &                      \\
    \hline
    Tag di ricerca        & Semplice      & Protezione bassa                    & Input                   &                      \\
    \hline
    Elenco Profili        & Composto      & Protezione bassa                    & Output                  & Non più di 5 profili \\
    \hline
\end{tabular}
\hfill \break

\textbf{GestioneProfili: Tabella Informazioni/Flusso}
\hfill \break

\begin{tabular} {|P{3cm}|P{3cm}|P{3cm}|P{3cm}|P{3cm}|}
    \hline
    \textbf{Informazione}           & \textbf{Tipo} & \textbf{Livello protezione/privacy} & \textbf{Input/Output} & \textbf{Vincoli} \\
    \hline
    Lista Profili                   & Composta      & Protezione media                    & Output                &                  \\
    \hline
    Identificativo Utente           & Semplice      & Protezione alta                     & Output                &                  \\
    \hline
    Identificativo Profilo corrente & Semplice      & Protezione alta                     & Output                &                  \\
    \hline
\end{tabular}
\hfill \break

\textbf{VisualizzaEvento: Tabella Informazioni/Flusso}
\hfill \break

\begin{tabular} {|P{3cm}|P{3cm}|P{3cm}|P{3cm}|P{3cm}|}
    \hline
    \textbf{Informazione}   & \textbf{Tipo} & \textbf{Livello protezione/privacy} & \textbf{Input/Output} & \textbf{Vincoli}                          \\
    \hline
    Identificativo Evento   & Semplice      & Protezione alta                     & Output                &                                           \\
    \hline
    Titolo Evento           & Semplice      & Protezione media                    & Input/Output          & Non più di 256 caratteri                  \\
    \hline
    Descrizione Evento      & Semplice      & Protezione media                    & Input/Output          & Non più di 1024 caratteri                 \\
    \hline
    Data e orario di inizio & Semplice      & Protezione media                    & Input/Output          & Deve essere precedente alla data di fine  \\
    \hline
    Data e orario di fine   & Semplice      & Protezione media                    & Input/Output          & Deve essere sucessiva alla data di inizio \\
    \hline
    Confermato              & Semplice      & Protezione media                    & Input/Output          &                                           \\
    \hline
    Immagini                & Composto      & Protezione media                    & Input/Output          &                                           \\
    \hline
\end{tabular}
\hfill \break

\textbf{AggiornaEvento: Tabella Informazioni/Flusso}
\hfill \break

\begin{tabular} {|P{3cm}|P{3cm}|P{3cm}|P{3cm}|P{3cm}|}
    \hline
    \textbf{Informazione} & \textbf{Tipo} & \textbf{Livello protezione/privacy} & \textbf{Input/Output} & \textbf{Vincoli}                                                        \\
    \hline
    Nome Cliente          & Semplice      & Protezione media                    & Input                 & Non più di 40 caratteri                                                 \\
    \hline
    Cognome Cliente       & semplice      & Protezione media                    & Input                 & Non più di 40 caratteri                                                 \\
    \hline
    Data di Nascita       & semplice      & Protezione media                    & Input                 & Deve avere più di 16 anni e data di nascita successiva al 1900          \\
    \hline
    Codice Fiscale        & semplice      & Protezione media                    & Input                 & Deve essere di 16 caratteri                                             \\
    \hline
    Email                 & semplice      & Protezione alta                     & Input                 & Deve essere di 256 caratteri e del formato giusto                       \\
    \hline
    Password              & semplice      & Protezione molto alta               & Input                 & Deve essere almeno di 8 caratteri, di cui uno alfabetico e uno numerico \\
    \hline
\end{tabular}
\hfill \break

\textbf{RecuperaImmagini: Tabella Informazioni/Flusso}
\hfill \break

\begin{tabular} {|P{3cm}|P{3cm}|P{3cm}|P{3cm}|P{3cm}|}
    \hline
    \textbf{Informazione} & \textbf{Tipo} & \textbf{Livello protezione/privacy} & \textbf{Input/Output} & \textbf{Vincoli}                                                        \\
    \hline
    Nome Cliente          & Semplice      & Protezione media                    & Input                 & Non più di 40 caratteri                                                 \\
    \hline
    Cognome Cliente       & semplice      & Protezione media                    & Input                 & Non più di 40 caratteri                                                 \\
    \hline
    Data di Nascita       & semplice      & Protezione media                    & Input                 & Deve avere più di 16 anni e data di nascita successiva al 1900          \\
    \hline
    Codice Fiscale        & semplice      & Protezione media                    & Input                 & Deve essere di 16 caratteri                                             \\
    \hline
    Email                 & semplice      & Protezione alta                     & Input                 & Deve essere di 256 caratteri e del formato giusto                       \\
    \hline
    Password              & semplice      & Protezione molto alta               & Input                 & Deve essere almeno di 8 caratteri, di cui uno alfabetico e uno numerico \\
    \hline
\end{tabular}
\hfill \break

\textbf{ScritturaLog: Tabella Informazioni/Flusso}
\hfill \break

\begin{tabular} {|P{3cm}|P{3cm}|P{3cm}|P{3cm}|P{3cm}|}
    \hline
    \textbf{Informazione}   & \textbf{Tipo} & \textbf{Livello protezione/privacy} & \textbf{Input/Output} & \textbf{Vincoli}        \\
    \hline
    Data                    & semplice      & Protezione media                    & Input                 & Non più di 40 caratteri \\
    \hline
    Ora                     & semplice      & Protezione media                    & Input                 & Non più di 40 caratteri \\
    \hline
    Attore                  & semplice      & Protezione alta                     & Input                 & Non più di 20 caratteri \\
    \hline
    Identificativo Farmacia & semplice      & Protezione alta                     & Input                 & Non più di 20 caratteri \\
    \hline
    Operazione Eseguita     & composto      & Protezione alta                     & Input                 &                         \\
    \hline
    Evento                  & composto      & Protezione molto alta               & Input                 &                         \\
    \hline
\end{tabular}

\newpage
\subsubsection{Analisi Documento dei Requisiti: Analisi dei Vincoli}
\hfill \break

\textbf{Tabella Vincoli}
\hfill \break


\begin{tabular} {|P{4cm}|P{2.5cm}|P{3.5cm}|P{6cm}|}
    \hline
    \textbf{Requisito}                  & \textbf{Categorie} & \textbf{Impatto}                                                         & \textbf{Funzionalità}                                                     \\
    \hline
    Semplicità dell'interfaccia         & Usabilità          & Intuitività di utilizzo                                                  & GestioneFarmacia, Registrazione, RicercaFarmaci, Login, NuovaPrenotazione \\
    \hline
    Velocità della ricerca dei dati     & Tempo di Risposta  & Maggiore reattività                                                      & GestioneFarmacia, Registrazione, RicercaFarmaci, Login, NuovaPrenotazione \\
    \hline
    Velocità di memorizzazione dei dati & Tempo di Risposta  & Maggiore reattività                                                      & GestioneFarmacia, Registrazione, Login, NuovaPrenotazione                 \\
    \hline
    Controllo Accessi                   & Sicurezza          & Peggiorano tempo di risposta e usabilità, migliorano la privacy dei dati & GestioneFarmacia, NuovaPrenotazione                                       \\
    \hline
    Protezione dei Dati                 & Sicurezza          & Peggiorano tempo di risposta, migliorano la privacy dei dati             & GestioneFarmacia, Registrazione, RicercaFarmaci, Login, NuovaPrenotazione \\
    \hline
\end{tabular}

\newpage

\subsubsection{Analisi Documento dei Requisiti: Analisi delle Interazioni}
\hfill \break

\textbf{Tabella Maschere}
\hfill \break

\begin{tabular} {|P{5cm}|P{7cm}|P{4cm}|}
    \hline
    \textbf{Maschera}          & \textbf{Informazioni}                                                                                     & \textbf{Funzionalità} \\
    \hline
    Home Gestione              & messaggio di benvenuto e scelta della funzionalità                                                        & GestioneFarmacia      \\
    \hline
    View Login                 & email, password                                                                                           & Login                 \\
    \hline
    View Prenotazioni          & lista prenotazioni                                                                                        & GestioneFarmacia      \\
    \hline
    View ResocontoUtenti       & nome cliente, cognome cliente, codice fiscale cliente, stato cliente                                      & GestioneFarmacia      \\
    \hline
    View VerificaIdentità      & nome cliente, cognome cliente, codice fiscale cliente                                                     & VeriticaIdentità      \\
    \hline
    View Farmaci               & lista farmaci                                                                                             & GestioneFarmacia      \\
    \hline
    Home Servizio              & messaggio di benvenuto, nome farmaco, località utente, lista farmacie pertinenti                          & RicercaFarmaci        \\
    \hline
    View Registrazione         & nome cliente, cognome cliente, data di nascita, codice fiscale, email, password                           & Registrazione         \\
    \hline
    View NuovaPrenotazione     & data invio, ora invio, data prenotazione, elenco farmaci, identificativo farmacia, identificativo cliente & NuovaPrenotazione     \\
    \hline
    View PrenotazioniPersonali & lista prenotazioni                                                                                        & ListaPrenotazioni     \\
    \hline
\end{tabular}
\hfill \break
\hfill \break

\textbf{Tabella Sistemi Esterni}
\hfill \break

\begin{tabular} {|P{2.5cm}|P{4cm}|P{4cm}|P{4.5cm}|}
    \hline
    \textbf{Sistema}              & \textbf{Descrizione}                                          & \textbf{Protocollo di Interazione}                                                & \textbf{Livello di Sicurezza}                                    \\
    \hline
    Gestione \linebreak Magazzino & Sistema che si occupa della gestione dei farmaci in magazzino & Mette a disposizione il modello publisher-subscriber per notificare aggiornamenti & Medio livello di sicurezza perchè protegge i dati della farmacia \\
    \hline
\end{tabular}
\hfill \break

\newpage
\subsubsection{Analisi Ruoli e Responsabilità}
\hfill \break

\textbf{Tabella Ruoli}
\hfill \break

\begin{tabular} {|P{2.5cm}|P{3cm}|P{4cm}|P{3cm}|P{2.5cm}|}
    \hline
    \textbf{Ruolo}                                                          & \textbf{Responsabilità}                                       & \textbf{Maschere} & \textbf{Riservatezza} & \textbf{Numerosità} \\
    \hline
    Farmacista                                                              & Gestione di tutte le informazioni relative agli utenti e alle
    prenotazioni di una farmacia                                            & Home Gestione, View Login, View
    Prenotazioni, View ResocontoUtenti, View VerificaIdentità, View Farmaci &
    È richiesto un alto grado di riservatezza                               & Massimo 10 farmacisti per ogni
    farmacia                                                                                                                                                                                                  \\
    \hline
    Cliente                                                                 & Ricerca di un farmaco senza necessità di login                & Home Servizio,
    View Login, View Registrazione                                          & È richiesto un medio grado di
    riservatezza                                                            & Illimitati                                                                                                                      \\
    \hline
    % roba molto brutta ma non so perchè non va a capo automaticamente questo
    ClienteRegi-\linebreak strato                                           & Ricerca e prenotazione di farmaci presso una farmacia
                                                                            & Home Servizio, View NuovaPrenotazione, View
    PrenotazioniPersonali                                                   & È richiesto un alto grado di riservatezza                     &
    Illimitati                                                                                                                                                                                                \\
    \hline
\end{tabular}
\hfill \break
\hfill \break

\textbf{Farmacista: Tabella Ruolo-Informazioni}
\hfill \break

\begin{tabular} {|P{7cm}|P{7cm}|}
    \hline
    \textbf{Informazione} & \textbf{Tipo di Accesso} \\
    \hline
    Nome Cliente          & Lettura                  \\
    \hline
    Cognome Cliente       & Lettura                  \\
    \hline
    Codice Fiscale        & Lettura                  \\
    \hline
    Stato Cliente         & Lettura/Scrittura        \\
    \hline
    Lista Farmaci         & Lettura/Scrittura        \\
    \hline
    Lista Prenotazioni    & Lettura/Scrittura        \\
    \hline
\end{tabular}
\hfill \break
\hfill \break

\textbf{ClienteRegistrato: Tabella Ruolo-Informazioni}
\hfill \break

\begin{tabular} {|P{7cm}|P{7cm}|}
    \hline
    \textbf{Informazione}     & \textbf{Tipo di Accesso} \\
    \hline
    Nome Cliente              & Lettura/Scrittura        \\
    \hline
    Cognome Cliente           & Lettura/Scrittura        \\
    \hline
    Data di Nascita           & Lettura                  \\
    \hline
    Codice Fiscale            & Lettura                  \\
    \hline
    Email                     & Lettura/Scrittura        \\
    \hline
    Password                  & Lettura/Scrittura        \\
    \hline
    Nome Farmaco              & Scrittura                \\
    \hline
    Località Utente           & Lettura                  \\
    \hline
    Lista Farmacie Pertinenti & Lettura                  \\
    \hline
    Data prenotazione         & Scrittura                \\
    \hline
    Elenco farmaci            & Scrittura                \\
    \hline
\end{tabular}
\hfill \break

\textbf{Cliente: Tabella Ruolo-Informazioni}
\hfill \break

\begin{tabular} {|P{7cm}|P{7cm}|}
    \hline
    \textbf{Informazione}     & \textbf{Tipo di Accesso} \\
    \hline
    Nome Farmaco              & Scrittura                \\
    \hline
    Località Utente           & Scrittura                \\
    \hline
    Lista Farmacie Pertinenti & Lettura                  \\
    \hline
\end{tabular}
\hfill \break
\hfill \break

\subsubsection{Scomposizione del Problema}
\hfill \break

\textbf{Tabella Scomposizione Funzionalità}
\hfill \break

\begin{tabular} {|P{7cm}|P{7cm}|}
    \hline
    \textbf{Funzionalità} & \textbf{Scomposizione}                                                     \\
    \hline
    GestioneFarmacia      & ResocontoFarmaci, ResocontoUtenti, ControlloPrenotazioni, VerificaIdentità \\
    \hline
    GestionePrenotazioni  & NuovaPrenotazione, ListaPrenotazioni                                       \\
    \hline
    ControlloPrenotazioni & ConfermaPrenotazione                                                       \\
    \hline
    ResocontoUtenti       & SospensioneUtenza                                                          \\
    \hline
\end{tabular}
\hfill \break

Non sono presenti legami di esclusione o di necessità tra le sotto-funzionalità del sistema.

\newpage
\subsubsection{Creazione Modello del Dominio}

Il seguente diagramma delle classi rappresenta la parte di modello del dominio relativa al sistema. \\

\begin{figure}[h!]
    \begin{center}
        \includegraphics[width=\textwidth]{ModelloDominio.jpg}
    \end{center}
\end{figure}
\hfill \break

\newpage
\subsubsection{Architettura Logica: Struttura}
\hfill \break

\textbf{Diagramma dei package}
\hfill \break

\begin{figure}[h!]
    \begin{center}
        \includegraphics[scale=0.6]{Diagrammi-Package.png}
    \end{center}
\end{figure}
\hfill \break

\textbf{Diagramma delle classi: Dominio}
\hfill \break

Non viene riportato il diagramma delle classi associato al package Dominio in quanto è il modello del dominio creato nella fase precedente.

\newpage

\textbf{Diagramma delle classi: InterfacciaGestioneAccesso \& GestioneAccesso}
\hfill \break

\begin{figure}[h!]
    \begin{center}
        \includegraphics[width=\textwidth]{Diagrammi-Gestione Accesso.jpg}
    \end{center}
\end{figure}
Tutte le ViewLogin contattano lo stesso Controller, che distingue il tipo di
credenziali da verificare in base ad un parametro che gli viene passato dalla
view che fa la richiesta
\hfill \break

\textbf{Diagramma delle classi: InterfacciaUtente \& RicercaFarmaci \& GestionePrenotazioni }

\begin{figure}[h!]
    \begin{center}
        \includegraphics[width=\textwidth]{Diagrammi-Utente.jpg}
    \end{center}
\end{figure}
HomeServizio fornisce la funzione di ricerca farmaco contattando direttamente
il controller. Inoltre per i clienti ad aver fatto il login espone anche le
funzionalità relative alle prenotazioni, come quella di crearne una nuova o di
elencare quelle effettuate.

\newpage
\textbf{Diagramma delle classi: InterfacciaGestioneFarmacia \& GestioneFarmacia}
\hfill \break

\begin{figure}[h!]
    \begin{center}
        \includegraphics[width=\textwidth]{Diagrammi-Farmacia.jpg}
    \end{center}
\end{figure}
L'InterfacciaFarmacia raccoglie tutte le funzionalità accessibili da
HomeGestione, le quali si relazionano con i rispettivi controller contenuti in
GestioneFarmacia.

\newpage

\subsubsection{Architettura Logica: Interazione}

In seguito saranno riportati i principali diagrammi di sequenza durante un normale utilizzo dell'applicazione.
\hfill \break

\textbf{Diagramma di Sequenza: Login Utente}

\begin{figure}[h!]
    \begin{center}
        \includegraphics[scale=0.5]{Interazione-LoginUtente.jpg}
    \end{center}
\end{figure}

\textbf{Diagramma di Sequenza: Login Farmacista}

\begin{figure}[h!]
    \begin{center}
        \includegraphics[scale=0.5]{Interazione-LoginFarmacista.jpg}
    \end{center}
\end{figure}
\hfill \break

\textbf{Diagramma di Sequenza: Nuova Prenotazione}

\begin{figure}[h!]
    \begin{center}
        \includegraphics[scale=0.5]{Interazione-NuovaPrenotazione.jpg}
    \end{center}
\end{figure}

\newpage

\textbf{Diagramma di Sequenza: Registrazione Utente}

\begin{figure}[h!]
    \begin{center}
        \includegraphics[scale=0.4]{Interazione-RegistrazioneUtente.jpg}
    \end{center}
\end{figure}
\hfill \break

\textbf{Diagramma di Sequenza: Conferma Prenotazione}

\begin{figure}[h!]
    \begin{center}
        \includegraphics[scale=0.45]{Interazione-ConfermaPrenotazione.jpg}
    \end{center}
\end{figure}
\newpage

\textbf{Diagramma di Sequenza: Ricerca Farmaco}

\begin{figure}[h!]
    \begin{center}
        \includegraphics[scale=0.5]{Interazione-RicercaFarmaco.jpg}
    \end{center}
\end{figure}
\hfill \break

\textbf{Diagramma di Sequenza: Sospensione Utenza}

\begin{figure}[h!]
    \begin{center}
        \includegraphics[scale=0.5]{Interazione-SospensioneUtenza.jpg}
    \end{center}
\end{figure}
\hfill \break

\newpage
\subsubsection{Architettura Logica: Comportamento}
\hfill \break

\textbf{Diagramma di Stato: Analizza Utente}
\hfill \break
Il seguente diagramma di stato illustra come vengono aggiornati gli stati dei
vari utenti in seguito a delle mancate conferme di prenotazione.

\begin{figure}[h!]
    \begin{center}
        \includegraphics[scale=0.5]{Comportamento.jpg}
    \end{center}
\end{figure}
\hfill \break

\newpage
\subsubsection{Piano di Lavoro}

I compiti sono stati divisi in base alle competenze di
ogni membro del gruppo come indicato nella tabella sottostante:
\hfill \break

\begin{tabular} {|P{5cm}|P{5cm}|P{5cm}|} % Qua cambiate a piacimento la larghezza
    \hline
    \textbf{Package}          & \textbf{Progetto}         & \textbf{Sviluppo}  \\
    \hline
    Dominio                   & Guerra,Palaferri,Romanini & Guerra             \\
    \hline
    Log                       & Guerra,Palaferri,Romanini & Guerra             \\
    \hline
    RicercaFarmaci            & Guerra,Palaferri,Romanini & Palaferri          \\
    \hline
    GestionePrenotazioni      & Guerra,Palaferri,Romanini & Palaferri,Romanini \\
    \hline
    GestioneAccesso           & Guerra,Palaferri,Romanini & Guerra,Romanini    \\
    \hline
    GestioneFarmacia          & Guerra,Palaferri,Romanini & Palaferri,Romanini \\
    \hline
    InterfacciaUtente         & Guerra,Palaferri,Romanini & Romanini           \\
    \hline
    InterfacciaGestioneAccsso & Guerra,Palaferri,Romanini & Guerra             \\
    \hline
    InterfacciaFarmacia       & Guerra,Palaferri,Romanini & Palaferri,Romanini \\
    \hline
\end{tabular}
\hfill \break

I tempi di rilascio sono i seguenti:
\begin{itemize}
    \item Progettazione entro due settimane dalla data odierna
    \item Sviluppo dei vai moduli con annessi test unitari entro una settimana dalla fine della fase di progettazione
    \item Integrazione e testing del sistema entro una settimana dalla fine dello sviluppo
\end{itemize}
\hfill \break

\textbf{Sviluppi Futuri}
\\

Il cliente ha richiesto la creazione di un applicativo mobile per sistemi
android e iOS, con l'obbiettivo di rendere il più pratico possibile l'utilizzo
del programma.
